\subsection{\tfcode{import files}}
\def\kapitelautor{Erik Ritschl}

This is the backend module for importing files into the currently active
gallery, but it also contains some frontend elements.

Its main function is \tfcode{import_files}, which takes either a
list of file paths or a \emph{Python dictionary} \cite{PythonDict}, where file paths are the keys and
lists of tags the values, as its argument. The former option corresponds to a
\emph{Direct file import'}, and the latter to an import via the more complex and
powerful \emph{import} mode (accessible via \emph{Start file import process}).

In both cases, the given data the is being iterated over, and a number of
procedures are executed for each file. First, a check is made to ensure that
the file actually exists. Then, a random unique ID gets generated, in form of
an UUID\index{UUID}. The file is then either moved or copied (a preference
which can be set in the settings) to the \emph{files} folder in the gallery,
using the UUID as its name, but retaining its file extension. This is necessary to avoid name
conflicts, since all files are located in the same directory. Finally, an entry
is made in the database, saving the file's UUID as well as its
original name. When a dictionary with tags was used as the argument, the tag relations
are also made here.

As mentioned earlier, there are also some frontend elements. These consist of
pop ups that inform the user of any errors that occurred during the import, as
well as a progress bar which shows how many files have yet to be imported.
There is also an informational pop up at the end that warns the user about any
files that have the same name. Same name files are supported by OctoTagger in
principle, but there are some not easily avoidable inconsistencies that come
with them. It was therefore decided to warn the user about the potential issues
and urging him or her to change the file names.

In the case that users rename a file via the \tfcode{RenameItem} function in \tfcode{octotagger.py}, the method \tfcode{rename_file} from this module is called. It takes the new name, extracts its file extension, and updates the extension of the original file in the \emph{files} directory. This was at first thought not to be necessary, since most operating systems use the extension of the \emph{link}, and not the \emph{target} file in order to determine with which application it should be opened. However as it turns out, Microsoft changed this for some reason starting with Windows 8. That's why the link and target both need to share the same file extension.