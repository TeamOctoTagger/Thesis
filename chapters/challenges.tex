\section{wxWidgets}

Early during development the team struggled with wxWidgets. It seemed far to
difficult and time consuming creating even the simplest widgets. This was
caused by the sparse understanding of wxWidgets' methodology the team had at
that point in time.

But this misunderstanding was not to last, as every member of the team is a
quick learner; realizing that wxWidgets provides building blocks instead of
finished widgets took only a few days. Growing comfortable with the API and
learning the tricks of wxWidgets both helped with not only solving this problem
but also with speeding up development.

\section{Windows}

OctoTagger was planned to be cross-platform from the start, but the team
already knew that symlinks are not as easy on Windows as they are on other
platforms. This is the reason why one team member --- Christoph Führer --- was
assigned with the special task of finding a solution to creating symlinks on
Windows.

Throughout the development solutions kept being discovered at a steady pace.
Each one of them has been analysed, understood and tested with the product,
but, unfortunately, none of the solutions worked. The general problem seems to
be that all the answers and solutions discovered are outdated and no longer
apply.

%TODO cref not working
The only option left was to create a new and working solution to this problem.
Hints, technologies and various code snippets kept coming up during the
research, though never as concrete as the other solutions, and were finally
used to implement \emph{pywinlink}. For more details on how it was implemented
see \cref{subsec:mod:pywinlink}.

\section{Mac OS X}

It was assumed from the start that there would be no large differences between
Ubuntu and Mac OS X regarding implementation. Which was true for the most part.
However, there were still some challenges. One of the biggest issues was
that there had been problems with installing \emph{wxPython}  under the newest 
version of OS X, because Apple apparently changed the structure of installer packages,
and the installer offered by \url{http://wxpython.org/} is out of date. The only solution to this problem was to manually repackage the source files in a way that allowed the software to be installed, by following the instructions found on a blog entry \cite{blogentry}. This new package is distributed on our website, so user do not have to go through the same process.

Aside from some minor GUI and behaviour differences for which the code had to be adjusted slightly, there was also the issue of creating a shortcut, which is described in more detail in \cref{sub:mod:install}. But the biggest hurdle was definitely testing, since none of our team members own an Apple PC. Computers had to be borrowed from friends and family, and where not available all the time. That is why bugs on OS X were always discovered relatively late. But all in all, we were right with our assumption that the differences between Linux and OS X aren't that large, and there were a lot less problems than with Windows.
