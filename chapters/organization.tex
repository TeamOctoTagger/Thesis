\section{Development Methodology}
\def \kapitelautor {Clemens Stadlbauer}

The software development methodology with which this project is developed takes
most of its characteristics from Scrum but also --- even though it seems
contradicting --- some from the Waterfall methodology. Agility and iterative
prototyping is provided by Scrum to ease change management during development.
However, the solid foundation provided by Waterfall provides stability and
helps to focus the development efforts.

% FIXME source
With Waterfall a project is split into these phases: \emph{Pre-Project},
\emph{Rough Planning}, \emph{Fine Planning}, \emph{Development},
\emph{Testing}, \emph{Post-Project}. Each phase strictly requires the previous
phase to be completed.
Due to the school requiring certain documents like an environmental, risk and
requirements analysis, which are usually present in the Waterfall methodology,
it was decided for this project to use the phases as described before but
without the exclusivity. In effect this means that all work during the
beginning of the project roughly fell under \emph{Planning}, all work during
the middle under \emph{Development}, etc. This way a clear, evolving goal is
available for the entire duration of the project.

% FIXME source
In stark contrast to Waterfall, a project utilizing Scrum is split into several
\emph{Sprints} which have a fixed timespan of around 1--2
weeks\footnote{\emph{Scrum} additionally requires \emph{Backlogs} which have
been omitted for brevity}. Each \emph{Sprint} has a general goal and several
tasks that need to be done. At the end of each \emph{Sprint} meetings are held
to review it and plan the next one.
All further requirements, like strict meeting guidelines and special management
roles, have been deemed surplus for this project due to the small team size and
the strong time constraint.

\section{Team Roles}
\def \kapitelautor {Clemens Stadlbauer}

% TODO Why this team, which skills, which roles

Each of our team members has a specialty when it comes to development. In this
section these specialties will be laid out and explained. % TODO bad ending

\subsection{Christoph Führer}

TODO

\subsection{Julian Lorenz}

While Julian is not as good as the other team members when it comes to
programming, he makes up for that with a passion for visual design and
websites. For these reasons he was assigned with the Corporate Identity, our
website and the user manual.

\subsection{Erik Ritschl}

As the one who originally had the idea and the driving force behind this
project, Erik quickly became the project leader.

\subsection{Clemens Stadlbauer}
The team leader deputy designated to TODO.
