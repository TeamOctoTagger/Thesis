\subsection{\tfcode{install}}
\label{sub:mod:install}
\def\kapitelautor{Erik Ritschl}

This \emph{Python} script is used to install OctoTagger under Unix platforms. It contains three functions: one for executing \tfcode{setup.py} (which is responsible for creating the initial default gallery and the system database), and one each for Max OS X and Linux to generate a launcher file.

Ubuntu stores application launchers as \emph{.desktop} files  in \tfpath{\textasciitilde{}/.local/share/applications}, which are available to the entire system. \tfcode{install.py} generates one of these for OctoTagger. As a result, users searching for "OctoTagger" in their launcher menu will find an appropriate entry with the project logo as icon, which can also be pinned to the taskbar. Native integration has been fully achieved.

%TODO source osacompile https://developer.apple.com/library/mac/documentation/Darwin/Reference/ManPages/man1/osacompile.1.html
%TODO source AppleScript https://developer.apple.com/library/mac/documentation/AppleScript/Conceptual/AppleScriptX/AppleScriptX.html
%TODO source environmentvars https://en.wikibooks.org/wiki/Guide_to_Unix/Environment_Variables
Unfortunately, OS X made things harder. Generating clickable shortcuts that execute two lines of terminal commands in order to launch a desktop application turned out to be surprisingly cumbersome on the Apple operating system. The best way to do this seems to be with a \emph{.app} file, which represents a standalone executable application. In order to procedurally generate one with \emph{Python}, the terminal command line tool \tfcode{osacompile} was used, which turns an \emph{AppleScript} into a \emph{.app} application. The \emph{AppleScript} contained code to launch the application with \emph{Python}, and since the path of the source files depends on where \tfcode{install.py} is located on the system when run, it had to be generated. 

However, during testing it was realized that the available \emph{environment variables} in \emph{AppleScript} are different from those that are normally available in a terminal window, which caused the started \emph{Python} instance to be unable to locate some of the essential libraries such as \emph{wx} and \emph{PIL}.

As a consequence, \emph{AppleScript} launchers were scrapped. Now, a simple \emph{.command} file is generated (which is simply a script file that is opened via double click) that launches OctoTagger with \emph{Python}). The downside to this is that a terminal window is opened as well. Ideally, the whole application should be packaged into a \emph{.app} file, with all its files and folders. But this would require considerably more work and is simply outside the scope of this project.