\section{\ldots{}of OctoTagger}
\def \kapitelautor {Erik Ritschl}

OctoTagger, the school project, ends here. But OctoTagger, the file
organization software, goes on.

We know that there is still room for many improvements in performance and
usability. We also have a long list of planned features, that will greatly
enhance the user experience. The first step will be to get as many people as
possible to try out OctoTagger, and collect feedback from them. Based on that,
we will implement new features and change or optimize existing ones.

We also want to try and find new use cases (e.g. archiving) and target a
broader audience, which will hopefully lead to greater popularity and adaption
of OctoTagger.

\section{\ldots{}of pywinlink}
\def \kapitelautor {Erik Ritschl}

Pywinlink is a side product of OctoTagger, and was created out of necessity and
frustration with Window's handling of symbolic links. It makes sense to
continue it as a separate project. We want that future projects by other
developers don't get stuck at the same problem we did, and that is why the
module is published on \textit{GitHub} in a separate repository, free for
everyone to adapt and modify.

Pywinlink's potential is great, since we know that there is a great demand for
a solution to the problem it fixes. Therefore we plan on improving the module,
making it more stable and powerful, and possibly releasing versions of it in
languages other than Python.

\section{\ldots{}of Us}
\def \kapitelautor {Erik Ritschl}

While it is not yet clear how many of the team members want to continue with
the project, at least the project leader plans to do so.

In the long run, we see OctoTagger as a hobby side project, that we will
continue to improve until there is no longer a need for it. It will always be a
great thing to have in our \textit{curricula vitae}. Should the interest in
OctoTagger be great enough, we could also consider to accept donations. Other
forms of monetization are very unlikely, since OctoTagger is, and always will
be, completely free software.
