\subsection{ItemView}
\def\kapitelautor{Clemens Stadlbauer}

\subsubsection{Abstract}
This module lays out a list of items consisting of a thumbnail and a name in a
grid with room to grow off the bottom of the frame. The thumbnails are loaded
in a background thread to make the interface immediately available to the user.

\subsubsection{Attempts}
The original idea was to have a \tfcode{GridSizer} and a rough size for each
item. The number of items per row would have been calculated based on the size
of the containing frame. The problem with this attempt is that quite many
expensive operations and calculations would have to be done on every resize,
potentially resulting in a degraded user experience along with a cluttered
codebase. For this reason a simpler and more elegant solution was searched.

Due to a lack of knowledge on the available \tfcode{Sizers} in wxWidgets the
strategy was to basically try out every one and analyse it. There always were
some problems or limitations with the used \tfcode{Sizers}, until the
\tfcode{WrapSizer} was discovered.

The only problem with this sizer is that it grows off to the right by default
and there seemed to be no way to tell it to grow off the bottom. After trying
to limit the available vertical space of the sizer with no success, it was
placed inside a \tfcode{ScrolledWindow} and after figuring out how to properly
set the size of this window the solution was found.

\subsubsection{Solution} % TODO better title

\begin{sloppypar}
The \tfcode{ItemView} class extends from \tfcode{ScrolledWindow} and is
initialized to vertical scrolling (\tfcode{style=wx.VSCROLL}). Next
two sizers are created, where \tfcode{mainsizer}'s only purpose is to add a
border (\tfcode{wx.EXPAND | wx.LEFT | wx.TOP, border=10}) around
\tfcode{sizer}, which is initialized as a \tfcode{WrapSizer} with horizontal
stacking (\tfcode{wx.WrapSizer(wx.HORIZONTAL)}). Lastly the scroll-rate is set
to roughly half of an \tfcode{Item}'s dimensions for an appropriate scrolling
speed. \Cref{lst:mod:itemview} shows the complete constructor.
\end{sloppypar}

\begin{sloppypar}
The item is a simple panel with a vertical \tfcode{BoxSizer} containing an
image at the top and the name at the bottom. The text is horizontally centered
(\tfcode{wx.ALIGN_CENTER_HORIZONTAL}) and cut off with an ellipsis if it
doesn't fit in the provided horizontal space (\tfcode{wx.ST_NO_AUTORESIZE |
wx.ST_ELLIPSIZE_END}). \Cref{lst:mod:itemview:item} shows the complete
constructor.
\end{sloppypar}

The number of items displayed during regular use of the program can reach such
a high amount that the loading of the thumbnails can take multiple seconds
degrading the user experience in the process. To resolve this issue all
thumbnails are loaded in the background via the \tfcode{ThumbnailThread} class.
As shown in \cref{lst:mod:itemview:thumbnail} the thread is initialized with
the calling window (\tfcode{notify_window}), the path to the actual image
(\tfcode{path}) and the path to the generated thumbnail or \tfcode{None} if it
doesn't exist yet (\tfcode{thumb}).

Once the thread is started it proceeds to generate the thumbnail, if needed,
with thumbnail module (TODO cref). Once it is generated the thread then creates
an \tfcode{Image} from the thumbnail, resizes it to desired size and finally
signals the calling window that the thumbnail has loaded.

The calling window, which is always an \tfcode{Item}, takes the \tfcode{Image},
coverts it into a \tfcode{Bitmap} and displays it. The full code, including the
calling of the thread, is shown in \cref{lst:mod:itemview:item:thumbnail}.

\begin{listing}[p]
	\begin{minted}{python}
		class ItemView(wx.ScrollenWindow):
			def __init__(self, parent):
				super(ItemView, self).__init__(parent, style=wx.VSCROLL)

				self.mainsizer = wx.BoxSizer(wx.VERTICAL)
				self.sizer = wx.WrapSizer(wx.HORIZONTAL)

				self.mainsizer.Add(
					self.sizer,
					0,
					wx.EXPAND | wx.LEFT | wx.TOP,
					border=10,
				)

				self.SetSizer(self.mainsizer)

				self.SetScrollRate(
					THUMBNAIL_SIZE[0] / 2,
					THUMBNAIL_SIZE[1] / 2,
				)
	\end{minted}
	\caption{A stripped down version of \tfcode{ItemView}'s constructor}
	\label{lst:mod:itemview}
\end{listing}

\begin{listing}[p]
	\begin{minted}{python}
		class Item(wx.Panel):
			def __init__(self, parent, name):
				super(Item, self).__init__(parent)

				self.sizer = wx.BoxSizer(wx.VERTICAL)
				self.SetSizer(self.sizer)

				self.bitmap = wx.StaticBitmap(self)
				self.EVT_THUMBNAIL_LOAD, self.OnThumbnailLoad)
				self.LoadThumbnail()

				self.sizer.Add(self.bitmap)

				self.text = wx.StaticText(
					self,
					label=name,
					style=(
						wx.ALIGN_CENTER_HORIZONTAL |
						wx.ST_NO_AUTORESIZE |
						wx.ST_ELLIPSIZE_END
					),
				)

				self.text.SetMaxSize((THUMBNAIL_SIZE[0], -1))

				self.sizer.Add(self.text)
	\end{minted}
	\caption{A stripped down version of \tfcode{Item}'s constructor}
	\label{lst:mod:itemview:item}
\end{listing}

\begin{listing}[p]
	\begin{minted}{python}
		class _ThumbnailThread(threading.Thead):
			def __init__(self, notify_window, path, thumb):
				super(_ThumbnailThread, self).__init__()
				self._notify_window = notify_window
				self._path = path
				self._thumb = thumb

			def run(self):
				if self._thumb is None:
					self._thumb = thumbnail.get_thumbnail(self.path)
				image = wx.Image(self._thumb)
				image.Resize(THUMBNAIL_SIZE, (
					(THUMBNAIL_SIZE[0] - image.GetWidth()) / 2,
					(THUMBNAIL_SIZE[1] - image.GetHeight()) / 2,
				))
				wx.PostEvent(
					self._notify_window,
					ThumbnailLoadEvent(data=image),
				)
	\end{minted}
	\caption{A stripped down version of \tfcode{ThumbnailThread}'s
	implementation}
	\label{lst:mod:itemview:thumbnail}
\end{listing}

\begin{listing}[p]
	\begin{minted}{python}
		def OnThumbnailLoad(self, event):
			self.bitmap.SetBitmap(event.data.ConvertToBitmap())
			self.thumb_thread = None

		def LoadThumbnail(self):
			self.thumb_thread = _ThumbnailThread(self, self.path, self.thumb)
			self.thumb_thread.start()
	\end{minted}
	\caption{How an \tfcode{Item} handles a loaded thumbnail}
	\label{lst:mod:itemview:item:thumbnail}
\end{listing}
