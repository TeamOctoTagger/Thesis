\subsection{Modulename}
\def\kapitelautor{}
% TODO all of taggingview

\subsubsection{Abstract}

\subsubsection{Attempts}

Originally the image was displayed using a \emph{StaticBitmap} widgets centered
using sizers. Here the problem arose with transparent images which would not
display. This was solved by removing the StaticBitmap and instead scaling the
image to fit the panel and drawing it directly onto it using a \emph{DC}. The
next problem was that some images were displayed on a black background instead
of the gray of the panel, which came from scaling the image to fit the panel.
The solution was leave the image too small and drawing it in the center. Finally
some systems would not display the image at all, which was solved by
implementing a double buffer: every time the window is resized the buffer is
resized to fit and the image, with the appropriate background color, is drawn
into it and each time the system requests a redraw of the panel the buffer is
copied into it.

\subsubsection{Solution} % TODO better title

\paragraph{Drawing the image} has undergone some version throughout development,
but the current version uses a \emph{DC} to draw to an internal \emph{Bitmap}
that is painted to the screen on every paint event.

\paragraph{Image preloading} is done to improve the user experience since every
time a new picture has to be displayed it first needs to be loaded into memory
and then scaled to fit the frame perfectly. To improve loading times a ring
buffer is used. This buffer always has the current and the two neighboring
images loaded in full size, so that when the user changes the image it is
already loaded.

Every field in the buffer has a reference to a thread which loads the image.
When the image is changed the ring buffer is rotated to mark the next or
previous field, depending on user input, as the new current field. During this
the field opposite to the current is rotated over the ``edge'' of the ring
buffer, it moves from furthest back to furthest ahead or vice versa. This field
is invalidated, loaded with the information about the correct image and its
thread is started.
