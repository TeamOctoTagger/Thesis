\section{Development Methodology}
\def\kapitelautor{Clemens Stadlbauer}

The software development methodology with which this project was developed takes
most of its characteristics from Scrum but also --- even though it seems
contradictory --- some from the Waterfall methodology. Agility and iterative
prototyping is provided by Scrum to ease change management during development.
However, the solid foundation provided by Waterfall provides stability and
helps to focus the development efforts.

% FIXME source
With Waterfall a project is split into these phases: \emph{Pre-Project},
\emph{Rough Planning}, \emph{Fine Planning}, \emph{Development},
\emph{Testing}, \emph{Post-Project}. Each phase strictly requires the previous
phase to be completed.
Due to the school requiring certain documents like an environmental, risk and
requirements analysis, which are usually included in the Waterfall methodology,
it was decided for this project to use the phases as described before but
without the exclusivity. In effect this means that all work during the
beginning of the project roughly fell under \emph{Planning}, all work during
the midway under \emph{Development}, etc. This way a clear, evolving goal is
present for the entire duration of the project.

% FIXME source
In stark contrast to Waterfall, a project utilizing Scrum is split into several
\emph{Sprints} which have a fixed timespan of around 1--2
weeks. At the beginning of the project a \emph{Product Backlog} is created
containing all requirements for the product and at the beginning of each
\emph{Sprint} a general goal is chosen and some tasks from the \emph{Project
Backlog} are moved to the \emph{Sprint Backlog}. At the end of each
\emph{Sprint}, meetings are held to review it and plan the next one.  All
further requirements, like strict meeting guidelines and special management
roles, have been deemed dispensable for this project due to the small team size and
the strong time constraint.

\section{Team Roles}
\def\kapitelautor{Clemens Stadlbauer}

Erik Ritschl, who originally had the idea, % TODO cref
also is the project leader. His responsibilities were ensuring Linux
compatibility and implementing new features.

Clemens Stadlbauer is the project leader deputy, whose work was more focused on
the backend.

Christoph Führer is mainly focused the frontend, of which he has implemented
substantial parts.

Julian Lorenz, our designer, took on the challenges of creating the Corporate
Design, building the website, writing the user manual and testing the software.

This team was was put together as every member has a use for the software and their skills
cover every requirement without too much overlap.
