\subsection{import\_files}

This is the backend module for importing files into the currently active
gallery, but it also contains some frontend elements.

It only has a single function, \tfcode{import_files}, which takes either a
list of file paths or a python dictionary, where file paths are the keys and
lists of tags the values, as its argument. The former option corresponds to a
``Direct file import'', and the latter to an import via the more complex and
powerful import mode.

In both cases, the given data the is being iterated over, and a number of
procedures are executed for each file. First, a check is made to ensure that
the file actually exists. Then, a random unique ID gets generated, in form of
an UUID\index{UUID}. The file is then either moved or copied (a preference
which can be set in the settings) to the \tfcode{files} folder in the gallery,
using the UUID\index{UUID} as its name. This is necessary to avoid name
conflicts, since all files are located in the same directory. Finally, an entry
is made in the database, saving the file's UUID\index{UUID} as well as its
original name. When a dictionary with tags was passed over, the tag relations
are also made here.

As mentioned earlier, there are also some frontend elements. These consist of
pop ups that inform the user of any errors that occurred during the import, as
well as a progress bar which shows how many files have yet to be imported.
There is also an informational pop up at the end that warns the user about any
files that have the same name. Same name files supported by OctoTagger in
principal, but there are some not easily avoidable inconsistencies that come
with them. It was therefore decided to warn the user about the potential issues
and urging him or her to change the file names.
