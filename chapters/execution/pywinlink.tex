\subsection{Modulename}
\def\kapitelautor{}
% TODO all of pywinlink

\subsubsection{Abstract}

\subsubsection{Attempts}

\subsubsection{Solution} % TODO better title

\paragraph{The service} runs in the background with administrative privileges
and listens for a client to connect, reads the necessary information and
creates a new symlink according to this information. This is achieved with a
named pipe. Named pipes enable to programs to communicate with each other based
on a mutually known name. When the service is started it tries to create a
named pipe and crashes if it wasn't able to. This is the only time the service
can crash.

Now that the pipe is open the service listens. Once a connection is established
it tries to read the target path, the link path and whether or not a hardlink
instead of a softlink should be created, in this order. The paths are UTF-8
encoded and the third parameter is either \tfcode{"True"} or \tfcode{"False"}.
Now that the service has all the information it tries to create the symlink. In
the event of an error the client is notified about it along with an error
message, otherwise a success signal is sent back.

Finally, all writes to the pipe are flushed, the connection is closed and the
service listens for the next client.

\paragraph{The client} is a small module that provides just a \tfcode{symlink}
method. This method can be split into two general steps, connecting to the pipe
and communicating over the pipe.

To connect to a pipe the client has to take into consideration other clients
that might block it. Because of this, when the client notices that the pipe is
busy, it waits a small amount of time and then tries again.

Once the connection is established sends the two paths and the link type, each
time checking if writing to the pipe succeeded and throwing an error if not.
The two paths are sent as UTF-8 encoded strings of the absolute paths. Absolute
because the service is running in an unknown current working directory and
would interpret relative paths different to the client.

Finally, the client reads the return message from the service, if it is an
error message throws it, and closes the connection.
