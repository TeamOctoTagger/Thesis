\subsection{\tfcode{taggingview}}
\def\kapitelautor{Christoph Führer, Clemens Stadlbauer, Erik Ritschl}
% TODO all of taggingview

\subsubsection{Abstract}

The  \emph{tagging view} is started by double-clicking an image or by clicking on \emph{Tools -> Start tagging mode}. When done so, either the clicked image or the first file of the currently displayed items in the \emph{item view} is shown as large as possible in the \emph{main view}, which enables the users to look at each image in detail and inspect them one by one, in order to decide which tags apply.

\subsubsection{Attempts}

At first it was tried to implement this feature without the help of any external modules. After some development the first prototype of the \emph{tagging view} was created and it functioned well. The problem was that it only worked for \emph{JPG} files and changing it to work with \emph{PNG} and \emph{GIF} files as well was far more difficult than it seemed in the beginning. After discussing and researching a while the decision was made to rewrite large portions of the \emph{tagging view} and to use the external module \emph{PIL} \cite{PIL}. After resolving some issues with resizing the images correctly in order to fit the available space and everything seemed to work, it was noticed that some images were not shown in Windows.

Originally the image was displayed using a \tfcode{wx.StaticBitmap} widget centered
using sizers. While debugging, it was found out that the issue lied with the transparency value of some \emph{PNG} files. It was solved by removing the StaticBitmap and instead scaling the
image to fit the panel and drawing it directly onto it using a drawing control. The
next problem was that some images were displayed on a black background instead
of the gray of the panel, which came from scaling the image to fit the panel.
The solution was to mostly leave the image in its original size and to draw it in the center. Finally
some systems would not display the image at all, which was solved by
implementing a double buffer: every time the window is resized the buffer is
resized to fit and the image, with the appropriate background color, is drawn
into it and each time the system requests a redraw of the panel the buffer is
copied into it.

\subsubsection{Solution} % TODO better title

%TODO source
\paragraph{Drawing the image} has undergone several forms throughout development,
but the current version uses a variant of the \emph{wx.DC} class, which allows us to draw directly to an internal \emph{wx.Bitmap}, the widget for displaying bitmaps,
that is painted to the screen on every paint event.

\paragraph{Image preloading} is done to improve the user experience since every
time a new picture has to be displayed it first needs to be loaded into memory
and then scaled to fit the frame perfectly. To improve loading times a ring
buffer is used. This buffer always has the current and the two neighboring
images loaded in full size, so that when the user changes the image it is
already loaded.

Every field in the buffer has a reference to a thread which loads the image.
When the image is changed the ring buffer is rotated to mark the next or
previous field, depending on user input, as the new current field. During this
the field opposite to the current one is rotated over the ``edge'' of the ring
buffer, it moves from furthest back to furthest ahead or vice versa. This field
is invalidated, loaded with the information about the correct image and its
thread is started.

\paragraph{The topbar} is located at the top of the \tfcode{main view}. It contains arrow buttons for going left and right, which call the functions \tfcode{DisplayPrev} and \tfcode{DisplayNext} respectively, and displayed in its center is the name of the current file, as well as the position in the file queue, for example \emph{(4 / 11)}.

\paragraph{The rest} consists of helper functions and other features, such as 
\tfcode{GetItems} for returning all queued files, \tfcode{GetCurrentItem} for returning the currently displayed file's ID, and \tfcode{RemoveItem} for removing a file from the queue.
