\subsection{\tfcode{output}}
\def\kapitelautor{Clemens Stadlbauer}

\subsubsection{Abstract}

This module provides a multitude of methods for changing output folders.
Functionality spans from renaming a folder over changing the link type (TODO
cref to explanation) to handling file tag changes.

\subsubsection{Solution} % TODO better title

Every method has the same basic procedure. First, a list of all affected output
folders is acquired, then for each folder the current status is analysed and
finally the required changes are performed.

% TODO maybe go into detail how foreign files are handled?

\paragraph{Output folder placement}

is controlled with the \tfcode{move} and \tfcode{rename} methods that move or
rename an output folder, respectively.

\paragraph{Output folder structure} % TODO better title

is controlled with one of three methods. The first is \tfcode{change_link_type}
which switches an output folder from softlinks to hardlinks or vice versa. This
is the only method in this category that is applicable to both gallery output
folders and advanced output folders.

Specific to gallery output folders is \tfcode{change_gallery} which changes
the list of tags to be contained in the folder. Specific to advanced output
folders, however, is \tfcode{change_expression} which, as the name says,
changes the expression for the folder.

\paragraph{Output folders}

in general are handled by \tfcode{create_gallery}, \tfcode{delete_gallery} and
\tfcode{delete_folder}. The first is used to create the necessary folder
structure for gallery folders, while the latter are used to clean up output
folders requested to be deleted.

\paragraph{Files}

have many methods for editing. There is the simple
\tfcode{change} to assign or remove a tag from a file. Then there are
\tfcode{rename_file} and \tfcode{rename_tag} to change the displayed names of
existing files or tags. Finally, for removal, \tfcode{remove} and
\tfcode{delete_tag} are provided to delete files or tags, respectively.
