\subsection{\tfcode{thumbnail}}
\label{sub:mod:thumbnail}
\def\kapitelautor{Clemens Stadlbauer}

\subsubsection{Abstract}

This module keeps a register that maps mime-types to static images or file
handlers that generate the thumbnails. By default, only jpeg, png and gif
images are handled with a generic handler utilizing \tfcode{PIL}\index{PIL}
\cite{pil} to load, size, scale and finally save the generated thumbnail as a
png image. If an image has an unregistered mime-type a generic thumbnail will be
used.

\subsubsection{Attempts}

At first we used the original images but it quickly became apparent that this
wasn't the final solution as the interface would hang for one or two seconds on
large images. It was then decided to generate thumbnails of a fixed size and
save them on disk to improve loading times.

\subsubsection{Solution} % TODO better title

The modules has two functions, one for registering new handlers and another for
generating thumbnails.

The first, called \tfcode{register}, takes a mime-type and a handler. It first
checks whether a handler for the specified mime-type already exists, then if
the provided handler is either a valid path or a function. If both checks pass
it registers the new handler.

To actually generate the thumbnail the modules provides the
\tfcode{get_thumbnail} method which takes either a file id from the database or
a path. It then proceeds to check if a thumbnail has previously been generated.
If none was found it tries to generate a new one by first finding the
appropriate handler from the file's mime-type. If the handler is a static image
its path is returned, if it is a function, however, it is invoked with the
file's path and the desired thumbnail's path. The newly generated thumbnail is
then returned.

In addition three default handlers are registered for the \tfcode{image/jpeg},
\tfcode{image/png} and \tfcode{image/gif} mime-types as these are the most
common image types. The generic handler uses the \emph{Python Image
Library}\index{PIL} to load the original image, resize it to fit the desired
size, add padding to exactly fit this size and finally save it as a png file.
While jpeg would provide a smaller file size, it does not support transparency.

To see how and when this module is used see \cref{subsec:mod:itemview}.
