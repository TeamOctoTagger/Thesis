\subsection{\tfcode{itemview}}
\label{subsec:mod:itemview}
\def\kapitelautor{Clemens Stadlbauer}

\subsubsection{Abstract}
This module provides the \tfcode{ItemView} class which lays out a list of
\tfcode{Item}s consisting of a thumbnail and a name in a grid with room to
extend beyond the bottom of the frame. The thumbnails are loaded in a
background thread (\tfcode{_ThumbnailThread}) to make the interface immediately
available to the user.

\subsubsection{Attempts}
The original idea was to have a \tfcode{wx.GridSizer} \cite{wxGridSizer} and an
appropriate size for each item. The number of items per row would have been
calculated based on the size of the containing frame. The problem with this
attempt is that quite many expensive operations and calculations would have to
be done on every resize, potentially resulting in a degraded user experience
along with a cluttered codebase. For this reason a simpler and more elegant
solution was searched.

Due to a lack of knowledge on the available \tfcode{Sizers} in
wxPython\index{wxPython} the strategy was to basically try out every one and
analyse it. There always were some problems or limitations with the used
\tfcode{Sizers}, until the \tfcode{wx.WrapSizer} \cite{wxWrapSizer} was
discovered.

The only problem with this sizer is that it extends to the right by default
and there seemed to be no way to tell it to extend downwards. After trying
to limit the available vertical space of the sizer with no success, it was
placed inside a \tfcode{wxScrolledWindow} \cite{wxScrolledWindow} and after
figuring out how to properly set the size of this window the solution was found.

\subsubsection{Solution} % TODO better title

See \tfpath{Source Code/octotagger/src/itemview.py} on the included CD to 
follow along in the code while reading the explanation below.

\begin{sloppypar}
The \tfcode{ItemView} class extends from \tfcode{wx.ScrolledWindow} and is
initialized to vertical scrolling (\tfcode{style=wx.VSCROLL}). Next
two sizers are created, where \tfcode{mainsizer}'s only purpose is to add a
border (\tfcode{wx.EXPAND | wx.LEFT | wx.TOP, border=10}) around
\tfcode{sizer}, which is initialized as a \tfcode{wx.WrapSizer} with horizontal
stacking (\tfcode{wx.WrapSizer(wx.HORIZONTAL)}). Lastly the scroll-rate is set
to roughly half of an \tfcode{Item}'s dimensions for an appropriate scrolling
speed.
\end{sloppypar}

\begin{sloppypar}
The item is a simple panel with a vertical \tfcode{wx.BoxSizer}
\cite{wxBoxSizer} containing an image at the top and the name at the bottom. The
text is horizontally centered (\tfcode{wx.ALIGN_CENTER_HORIZONTAL}) and cut off
with an ellipsis if it doesn't fit in the provided horizontal space
(\tfcode{wx.ST_NO_AUTORESIZE | wx.ST_ELLIPSIZE_END} and
\tfcode{SetMaxSize((THUMBNAIL_SIZE[0], -1))}). Event listeners for mouse click
events are set up to delegate them to the \tfcode{ItemView}.
\end{sloppypar}

Upon setting a list of files for \tfcode{ItemView} to display, a number of
steps are taken. First the list is split into several types of files based on
information about each element. If the element is of type \tfcode{int} it
refers to the id of a file in the database. If it is a string it can be an
external file, an external folder or, if the elements begins with
``GALLERYFOLDER'', a gallery folder. Next these sublists are sorted
alphabetically and case insensitive. Lastly they are brought back together in
the order of gallery folders first, then external folders, then external files
and in the end imported files. After all that the new and sorted list is
iterated once again, this time creating a new \tfcode{Item} for each element
with the respective display name, file location and thumbnail.

The number of items displayed during regular use of the program can reach such
a high amount that the loading of the thumbnails can take several seconds
degrading the user experience in the process. To resolve this issue all
thumbnails are loaded in the background via the \tfcode{ThumbnailThread} class.
The thread is initialized with the calling window (\tfcode{notify_window}), the
path to the actual image (\tfcode{path}) and the path to the generated
thumbnail or \tfcode{None} if it doesn't exist yet (\tfcode{thumb}).

Once the thread is started it proceeds to generate the thumbnail, if needed,
with thumbnail module (see \cref{sub:mod:thumbnail}). Once it is generated the
thread then creates an \tfcode{Image} from the thumbnail, resizes it to desired
size and finally signals the calling window that the thumbnail has loaded. The
calling window, which is always an \tfcode{Item}, takes the \tfcode{wx.Image}
\cite{wxImage}, coverts it into a \tfcode{wx.Bitmap} \cite{wxBitmap} and
displays it.
