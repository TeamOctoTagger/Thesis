\subsection{expression}
\def\kapitelautor{Clemens Stadlbauer}

\subsubsection{Abstract}

This module's job is to convert an expression (TODO cref) to its corresponding
SQL query. It does this by first splitting the expression into segments with
Python's \tfcode{Scanner} and then iterating over the segments ensuring
syntactical correctness and building the query in the process.

\subsubsection{Attempts}

Although they weren't multiple attempts during the implementation, there were
during planning.

The first idea was to represent an expression with special datatypes for each
possible token and combinations and groupings between tokens. While this would
have been the most expressive and structured solution it is sadly not possible
to implement in Python because Python has no support for annotated values due
to its lacking type system.

To circumvent this, an alternative solution would be to express values as a
list or dictionary carrying the annotation. The problem with this approach is
that the resulting structure would be nested far deeper than the original
expression and working with it would be needlessly complicated.

\subsubsection{Solution} % TODO better title

Shortly prior to implementing this module, Clemens, who's responsible for the
implementation, read about an undocumented feature inside Python's \tfcode{re}
module that provided exactly the functionality needed for the first step.
\tfcode{Scanner} is the class in question and it works by splitting a string
into parts based on a list of regular expressions and optionally passing the
part through a conversion function. For OctoTagger the parts don't need to be
converted but this feature is used to strip whitespace and error on unknown
tokens.

\begin{minted}{python}
    scanner = re.Scanner([
        (REG_NUM, identity),
        (REG_TAG, identity),
        (REG_OP, identity),
        (REG_GROUP, identity),
        (r'\s+', None),
        (r'', unknown),
    ])
\end{minted}

After a simple parenthesis count the tokens are step by step grouped together.
First the three tokens required for the numerical comparison (tag name, literal
``='' and number) are grouped into one token. Next all completed comparisons
are converted to a query and then, if present, inversions (read ``not'') are
applied.

At this point all comparisons are finished and only need to be combined. This
happens by first grouping together all tokens inside a group (enclosed by
parentheses). The last step is to split the tokens at ``or'' tokens (``/''),
pairing the tags within both sides with ``and'' and finally pairing the sides
with ``or''.

\paragraph{Additional methods}

In addition to converting an expression to a query, this module also provides
two methods map over tag names or ids inside an expression named
\tfcode{map_tag_name} and \tfcode{map_tag_id} respectively. What these
functions do is pass each tag name or id to a supplied callback function and
use the result of the callback as a replacement.

These functions are mainly used for converting an expression with tag names,
which the user enters, to an expression with tag ids, which is stored in the
database, and converting the other way.
% TODO cref where it is used?
