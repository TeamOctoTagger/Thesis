\section{Goals}

\section{Lessons Learned}
\def\kapitelautor{}

\subsection{Erik Ritschl}
\subsection{Clemens Stadlbauer}

I have learned quite a lot about the pitfalls a project can have while working
on OctoTagger. The most important aspect, to me at least, is the continuous
ensurance of quality. Defining a small set of absolutely mandatory goals at the
beginning that capture the vision of the project, creating a functional
prototype that tests all requirements and writing automated tests to prevent
regressions from happening.

Having a clear vision from the start discourages implementing features which do
not align with this vision. We did not have such a fixed vision and I noticed
from time to time newly implemented features that seemed a bit out of place
because everyone had their own idea of what the software should do. The source
code as well should be structured so that it represents the vision resulting
in a module tree that represents how the product functions. For OctoTagger we
did not define a module layout and now we have just a bunch of files that are
difficult to manage.

Each external technology a program uses can break in unexpected ways. For this
reason a prototype is crafted, to see if all technologies really do solve our
problems instead of introducing new ones. I always though a prototype only has
to test functionality but there is one more aspect it must have: it must fulfill
all the requirements of the project. For OctoTagger a prototype showing that
managing tagged files is possible was created but some assumptions, mainly about
symlinks on Windows, were made that have not been clearly defined in the
contract. Had we not made assumptions here we would have saved a lot of time.

During development bugs appeared, which we anticipated, but as we neared
project completion fixing bugs became more difficult due to everything being
fragile. Fixing one bug resulted in breaking multiple other, previously working
parts of the software. Most of these regression bugs were in the frontend
because none of us are experts in wxWidgets but it nontheless showed me the
importance of pinning down correct behaviour with a test.

\subsection{Christoph Führer}
\subsection{Julian Lorenz}

\section{Summary}
